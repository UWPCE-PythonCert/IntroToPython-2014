\documentclass{beamer}
%\usepackage[latin1]{inputenc}
\usetheme{Warsaw}
\title[Intro to Python: Week 1]{Introduction  to Python\\ Functions, Modules, and Sequences}
\author{Christopher Barker}
\institute{UW Continuing Education}
\date{October 8, 2013}

\usepackage{listings}
\usepackage{hyperref}

\begin{document}

% ---------------------------------------------
\begin{frame}
  \titlepage
\end{frame}

% ---------------------------------------------
\begin{frame}
\frametitle{Table of Contents}
%\tableofcontents[currentsection]
  \tableofcontents
\end{frame}


\section{Review/Questions}

% ---------------------------------------------
\begin{frame}[fragile]{Review of Previous Class}

\begin{itemize}
  \item Values and Types
  \item Expressions
  \item Intro to functions
\end{itemize}

\end{frame}


% ---------------------------------------------
\begin{frame}[fragile]{Lightning Talks}

\vfill
{\LARGE Lightning talks today:}

\vfill
{\Large
Jo-Anne Antoun 

\vfill
Omer Onen 

\vfill
Ryan Small

\vfill
Catherine Warren 
}
\vfill

\end{frame}


% ---------------------------------------------
\begin{frame}[fragile]{Homework review}

  \vfill
  {\Large Homework Questions? }

  \vfill
  {\Large My Solution}

  \vfill

\end{frame}

% ---------------------------------------------
\begin{frame}[fragile]{Stuff brought up by homework}

  \vfill
  {\Large Bytecode and \verb|*.pyc| }

  \vfill
  {\Large Please send me code:
    \begin{itemize}
      \item{enclosed in an email}
      \item{with your name at the beginning of the filename: \verb|chris_problem1.py|}
    \end{itemize}
  }

  \vfill
  {\Large PEP 8}

  \vfill
  {\Large Repeating variable names in nested loops}


\end{frame}


\section{Quick Intro to Basics}

\begin{frame}[fragile]{Basics}

\vfill
{\Large It turns out you can't really do much at all without at least a container type, conditionals and looping...}
\vfill

\end{frame}

\begin{frame}[fragile]{if}

{\Large When you need to make a decision:}

\begin{verbatim}
if x > 500:
    print "x is big!"
else:
    print "x is small"
\end{verbatim}

\end{frame}


\begin{frame}[fragile]{lists}

\vfill
{\Large A way to store a bunch of stuff in order}

\vfill
{\large ``array'' in other languages}

\vfill
\begin{verbatim}
a_list = [2,3,5,9]

a_list_of_strings = ['this', 'that', 'the', 'other']
\end{verbatim}

\vfill

\end{frame}

\begin{frame}[fragile]{tuples}

\vfill
{\Large Another way to store an ordered list of things}

\vfill
\begin{verbatim}
a_tuple = (2,3,4,5)

a_list_of_strings = ('this', 'that', 'the', 'other')
\end{verbatim}

\vfill
{\Large Often interchangeable with lists, but not always...}

\end{frame}


\begin{frame}[fragile]{for}

{\Large When you need to do something to everything in a sequence}

\vfill
\begin{verbatim}
>> a_list = [2,3,5,9]

>> for item in a_list:
>>     print item
2
3
5
9
\end{verbatim}

\end{frame}

\begin{frame}[fragile]{range and for}

{\Large When you need to do something a set number of times}

\vfill
\begin{verbatim}
>>> range(4)
[0, 1, 2, 3]
>>> for i in range(6):
...     print "*",
... 
* * * * * *
>>> 
\end{verbatim}

\end{frame}

\begin{frame}[fragile]{intricacies}

\vfill
{\Large This is enough to get you started.}

\vfill
{\Large Each of these have intricacies special to python}

\vfill
{\Large We'll get to those over the next couple classes}

\vfill

\end{frame}

%%%%%%%%%%%%%%%%%%%%%%
\section{More on Functions}
%%%%%%%%%%%%%%%%%%%%%%

\begin{frame}[fragile]{Functions: review}

{\Large Defining a function:}

\begin{verbatim}
def fun(x, y):
    z = x+y
    return z
\end{verbatim}

{\Large x, y, z are local names}

\end{frame}


\begin{frame}[fragile]{Functions: local vs. global}

\begin{verbatim}
x = 32
def fun(y, z):
    print x, y, z

fun(3,4)

32 3 4
\end{verbatim}
{\large x is global, y and z local}

\vfill
{\Large Use global variables mostly for constants}

\end{frame}


%----------------------------------
\begin{frame}[fragile]{Recursion}

\vfill
{\LargeRecursion is calling a function from itself.}

\vfill
{\LargeMax stack depth, function call overhead.}

\vfill
{\LargeBecause of these two(?), recursion isn't used {\bf that} often in Python.}

\vfill
(demo: factorial)
\end{frame}

%----------------------------------
\begin{frame}[fragile]{Tuple Unpacking}

{\Large Remember: \verb| x,y = 3,4| ?}

\vfill
{\Large Really ``tuple unpacking'': \verb| (x, y) = (3, 4)|}

\vfill
{\Large This works in function arguments, too:}

\begin{verbatim}
>>> def a_fun( (a, b), (c, d) ):
...     print a, b, c, d
... 
>>> t, u = (3,4), (5,6)
>>> 
>>> a_fun(t, u)
3 4 5 6
\end{verbatim}
(demo)
\end{frame}


%----------------------------------
\begin{frame}[fragile]{Lab: more with functions}

{\Large write a function that:}
\begin{itemize}
  \item copnutes the distance between two points:\\
        dist = sqrt( (x1-x2)**2 + (y1-y2)**2 )\\
        using tuple unpacking...
  \item Take some code with functions, add this to each function:\\
        \verb|print locals()|
  \item Computes the Fiboacci series with a recursive function:\\
  f(1) = 1\\
  f(n) = f(n-1) + f(n-2)\\
  0, 1, 1, 2, 3, 5, 8, 13, 21, ...
\end{itemize}

\end{frame}

%-------------------------------
\begin{frame}[fragile]{Lightning Talks}

\vfill
{\LARGE Lightning Talks:}

\vfill
{\Large Jo-Anne Antoun }

\vfill
{\Large Omer Onen }

\vfill
\end{frame}



%##########################
\section{Conditionals}

%-------------------------------
\begin{frame}[fragile]{if}

{\Large Making Decisions...}
\begin{verbatim}
if a:
    print 'a'
elif b:
    print 'b'
elif c:
    print 'c'
else:
    print 'that was unexpected'
\end{verbatim}

\end{frame}


%-------------------------------
\begin{frame}[fragile]{if}

{\Large Making Decisions...}
\begin{verbatim}
if a:
    print 'a'
elif b:
    print 'b'

## versus...

if a:
    print 'a'
if b:
    print 'b'
\end{verbatim}

\end{frame}

%-------------------------------
\begin{frame}[fragile]{switch?}

\vfill
{\Large No switch/case in Python}

\vfill
{\Large use \verb|if..elif..elif..else|}

\vfill

(or a dictionary, or subclassing....)
\end{frame}

%%-------------------------------
%\begin{frame}[fragile]{LAB}
%
%\begin{itemize}
%
%
%\item some if .. elif, etc. examples
%
%\end{itemize}
%
%\end{frame}

\section{Boolean Expressions}

% ---------------------------------------------
\begin{frame}[fragile]{Truthiness}

{\Large What is true or false in Python?}

\begin{itemize}
  \item The Booleans: \verb+True+ and \verb+False+
  \item ``Something or Nothing''
\end{itemize}

{\small \url{http://mail.python.org/pipermail/python-dev/2002-April/022107.html} }

\end{frame}

%-------------------------------
\begin{frame}[fragile]{Truthiness}

{\Large Determining Truthiness:}

\vfill
{\Large \verb+bool(something)+ }

\vfill


\end{frame}

% ---------------------------------------------
\begin{frame}[fragile]{Boolean Expressions}

{\Large \verb+False+ }

\begin{itemize}
  \item \verb+None+
  \item \verb+False+
  \item zero of any numeric type, for example, \verb+ 0, 0L, 0.0, 0j+.
  \item any empty sequence, for example, \verb+ '', (), [] +.
  \item any empty mapping, for example, \verb+{}+.
  \item instances of user-defined classes, if the class defines a
        \verb+__nonzero__() or __len__()+ method, when that method
        returns the integer zero or bool value \verb+False+.
\end{itemize}

\url{http://docs.python.org/library/stdtypes.html}

\end{frame}

% ---------------------------------------------
\begin{frame}[fragile]{Boolean Expressions}

{ \LargeAvoid: }

\vspace{0.1in}
\verb+if xx == True:+

\vfill
{ \LargeUse: }

\vspace{0.1in}
\verb+if xx:+

\vfill
\end{frame}

% ---------------------------------------------
\begin{frame}[fragile]{Boolean Expressions}

{\Large ``Shortcutting''}

\begin{verbatim}
                  if x is false, 
x or y               return y,
                     else return x

                  if x is false,
x and y               return  x
                      else return y

                  if x is false,
not x               return True,
                    else return False 
\end{verbatim}

\end{frame} 

% ---------------------------------------------
\begin{frame}[fragile]{Boolean Expressions}

{\Large Stringing them together}

\begin{verbatim}
 a or b or c or d

a and b and c and d  
\end{verbatim}

{\Large The first value that defines the result is returned}

\vfill
(demo)
\end{frame}


%---------------------------------------------
\begin{frame}[fragile]{Boolean returns}

{\Large From CodingBat}
\vfill
\begin{verbatim}
def sleep_in(weekday, vacation):
    if weekday == True and vacation == False:
        return False
    else:
        return True
\end{verbatim}

\end{frame}


%---------------------------------------------
\begin{frame}[fragile]{Boolean returns}

{\Large From CodingBat}

%\begin{verbatim}
%def makes10(a, b):
%    return a == 10 or b == 10 or a+b == 10
%\end{verbatim}

\begin{verbatim}
def sleep_in(weekday, vacation):
    return not (weekday == True and vacation == False)
\end{verbatim}

or

\begin{verbatim}
def sleep_in(weekday, vacation):
    return (not weekday) or vacation
\end{verbatim}


\end{frame}


% -------------------------------------------
\begin{frame}[fragile]{bools are ints?}

{\Large bool types are subclasses of integer}

\begin{verbatim}
In [1]: True == 1
Out[1]: True

In [2]: False == 0
Out[2]: True  
\end{verbatim}

{\Large It gets weirder! }

\begin{verbatim}
In [6]: 3 + True
Out[6]: 4
\end{verbatim}

(demo)

\end{frame}

%-------------------------------
\begin{frame}[fragile]{Conditional expression}

{\large A common idiom:}
\begin{verbatim}
if something:
    x = a_value
else:
    x = another_value
\end{verbatim}
\vfill
{\large Also, other languages have a ``ternary operator''}\\
\hspace{0.2in}(C family: \verb|result = a > b ? x : y ;|)

\vfill
{ \Large \verb|y = 5 if x > 2 else 3| }

\vfill
{\large PEP 308:}
(http://www.python.org/dev/peps/pep-0308/)

\end{frame}



%-------------------------------
\begin{frame}[fragile]{LAB}

\begin{itemize}
  \item Look up the \verb+%+ operator. What do these do?\\
    \verb| 10 % 7 == 3 | \\
    \verb| 14 % 7 == 0 |
  \item  Write a program that prints the numbers from 1 to 100 inclusive.
But for multiples of three print ``Fizz'' instead of the number and for the
multiples of five print ``Buzz''. For numbers which are multiples of both three
and five print ``FizzBuzz'' instead.
\end{itemize}

\end{frame}

%-------------------------------
\begin{frame}[fragile]{LAB}

\vfill
{\large Re-write a couple CodingBat exercises, using a conditional expression}


\vfill
{\large Re-write a couple CodingBat exercises, returning the direct boolean results}

\vfill
{\large (use whichever you like, or the ones in: \verb|code/codingbat.rst|)}

\end{frame}

%-------------------------------
\begin{frame}[fragile]{Lightning Talks}

{\LARGE Lightning Talks:}

\vfill
Ryan Small

\vfill
Catherine Warren 


\end{frame}


%%%%%%%%%%%%%%%%%%%%%%%%%%%%%%%%
\section{Code structure, modules, and namespaces}
%%%%%%%%%%%%%%%%%%%%%%%%%%%%%%%%

% ---------------------------------------------
\begin{frame}[fragile]{Code Structure}

{\Large Python is all about namespaces --  the ``dots'' }

\vspace{0.2in}
\verb+name.another_name+
\vspace{0.2in}

the ``dot'' indicates looking for a name in the namespace of the given object.

could be:

\begin{itemize}
\item name in a module
\item module in a package
\item attribute of an object
\item method of an object
\end{itemize}

\end{frame}

% ---------------------------------------------
\begin{frame}[fragile]{indenting and blocks}

{\Large  Indenting determines blocks of code }

\vfill
\begin{verbatim}
something:
    some code
    some more code
    another block:
        code in 
        that block
\end{verbatim}

\vfill
{\Large But you need the colon too...}

\end{frame}

% ---------------------------------------------
\begin{frame}[fragile]{indenting and blocks}

{\Large  You can put a one-liner after the colon:}

\vfill
\begin{verbatim}
In [167]: x = 12

In [168]: if x > 4: print x
12
\end{verbatim}

\vfill
{\Large Only do this if it makes it more readable...}

\end{frame}


\begin{frame}[fragile]{Spaces and Tabs}

{\Large  An indent can be:}
\begin{itemize}
  \item Any number of spaces
  \item A tab
  \item tabs and spaces:
    \begin{itemize}
      \item A tab is eight spaces (always!)
      \item Are they eight in your editor?
    \end{itemize}
\end{itemize}

\vfill
{\LARGE Use four spaces -- really!}

\vfill
(PEP 8)

\end{frame}


% ---------------------------------------------
\begin{frame}[fragile]{Spaces Elsewhere}

{\Large  Other than indenting -- space doesn't matter}

\vfill
\begin{verbatim}

x = 3*4+12/func(x,y,z)

x = 3*4 + 12 /   func (x,   y, z) 

\end{verbatim}

\vfill
{\Large Choose based on readability/coding style}

\vfill
\center{\LARGE PEP 8}

\end{frame}


% ---------------------------------------------
\begin{frame}[fragile]{Various Brackets}

{\Large Bracket types:}

\begin{itemize}
  \item parentheses \verb+( )+
    \begin{itemize}
      \item tuple literal: \verb+(1,2,3)+
      \item function call: \verb+fun( arg1, arg2 )+
      \item grouping: \verb| (a + b) * c |
    \end{itemize}
  \item square brackets \verb+[ ]+
    \begin{itemize}
      \item list literal: \verb+[1,2,3]+
      \item sequence indexing: \verb+a_string[4]+
    \end{itemize}
  \item curly brackets \verb+{ }+
    \begin{itemize}
      \item dictionary literal: \verb+{"this":3, "that":6}+
      \item (we'll get to those...)
    \end{itemize}
\end{itemize}

\end{frame}


% ************************************
\section {Sequences}

\begin{frame}[fragile]{Sequences}

{\Large Sequences are ordered collections of objects}

\vfill
{\Large They can be indexed, sliced, iterated over,...}

\vfill
{\Large They have a length:  \verb+len(sequence)+}

\vfill
{\Large Common sequences (Remember Duck Typing?):}

{\Large
\begin{itemize}
   \item strings
   \item tuples
   \item lists
\end{itemize}
}

\end{frame}

%-------------------------------
\begin{frame}[fragile]{Indexing}

{\Large square brackets for indexing: \verb+[]+}

\vfill
{\Large Indexing starts at zero}

\begin{verbatim}
In [98]: s = "this is a string"

In [99]: s[0]
Out[99]: 't'

In [100]: s[5]
Out[100]: 'i'
\end{verbatim}

\end{frame}

%-------------------------------
\begin{frame}[fragile]{Indexing}

{\Large Negative indexes count from the end}

\vfill
\begin{verbatim}
In [105]: s = "this is a string"

In [106]: s[-1]
Out[106]: 'g'

In [107]: s[-6]
Out[107]: 's'
\end{verbatim}

\end{frame}

%-------------------------------
\begin{frame}[fragile]{Slices}

{\Large Slicing: Pulling a range out of a sequence}

\begin{verbatim}
sequence[start:finish]  

indexes for which:

start <= i < finish
\end{verbatim}

\end{frame}

%-------------------------------
\begin{frame}[fragile]{Slices}
\begin{verbatim}
In [121]: s = "a bunch of words"
In [122]: s[2]
Out[122]: 'b'

In [123]: s[6]
Out[123]: 'h'

In [124]: s[2:6]
Out[124]: 'bunc'

In [125]: s[2:7]
Out[125]: 'bunch'
\end{verbatim}

\end{frame}


%-------------------------------
\begin{frame}[fragile]{Slices}

{\Large the indexes point to the spaces between the items}

\vfill
\begin{verbatim}
   X   X   X   X   X   X   X   X
 |   |   |   |   |   |   |   | 
 0   1   2   3   4   5   6   7
\end{verbatim}

\end{frame}

%-------------------------------
\begin{frame}[fragile]{Slices}

{\Large Slicing satisfies nifty properties:

\vfill
\begin{verbatim}
len( seq[a:b] ) == b - a

seq[a:b] + seq [b:c] == seq

\end{verbatim}

}

\end{frame}

% ------------------------------------------------
\begin{frame}[fragile]{Slicing vs. Indexing}

{\Large Indexing returns a single element}

\begin{verbatim}
In [86]: l
Out[86]: [0, 1, 2, 3, 4, 5, 6, 7, 8, 9]

In [87]: type(l)
Out[87]: list

In [88]: l[3]
Out[88]: 3

In [89]: type( l[3] )
Out[89]: int
\end{verbatim}
\end{frame}

% ------------------------------------------------
\begin{frame}[fragile]{Slicing vs. Indexing}

{\Large Unless it's a string:}

\begin{verbatim}
In [75]: s = "a string"

In [76]: s[3]
Out[76]: 't'

In [77]: type(s[3])
Out[77]: str
\end{verbatim}

\vfill
There is no single character type

\end{frame}


%-------------------------------
\begin{frame}[fragile]{Slicing vs. Indexing}

{\Large Slicing returns a sequence:}

\begin{verbatim}
In [68]: l
Out[68]: [0, 1, 2, 3, 4, 5, 6, 7, 8, 9]

In [69]: l[2:4]
Out[69]: [2, 3]
\end{verbatim}

Even if it's one element long

\begin{verbatim}
In [70]: l[2:3]
Out[70]: [2]

In [71]: type(l[2:3])
Out[71]: list
\end{verbatim}

\end{frame}

%-------------------------------
\begin{frame}[fragile]{Slicing vs. Indexing}

{\Large Indexing out of range produces an error}
\vfill
\begin{verbatim}
In [129]: s = "a bunch of words"
In [130]: s[17]
----> 1 s[17]
IndexError: string index out of range
\end{verbatim}

\vfill
{\Large Slicing just gives you what's there}

\begin{verbatim}
In [131]: s[10:20]
Out[131]: ' words'

In [132]: s[20:30]
Out[132]: ''
\end{verbatim}
(demo)
\end{frame}

% ---------------------------------------------
\begin{frame}[fragile]{Multiplying and slicing}

{\Large from CodingBat: Warmup-1 -- front3}

\begin{verbatim}
def front3(str):
  if len(str) < 3:
    return str+str+str
  else:
    return str[:3]+str[:3]+str[:3]
\end{verbatim}

{\Large or}

\begin{verbatim}
def front3(str):
    return str[:3] * 3
\end{verbatim}

\end{frame} 

% ---------------------------------------------
\begin{frame}[fragile]{Slicing}

{\Large from CodingBat: Warmup-1 -- \verb+missing_char+ }

\begin{verbatim}
def missing_char(str, n):
  front = str[0:n]
  l = len(str)-1
  back = str[n+1:l+1]
  return front + back
\end{verbatim}

\begin{verbatim}
def missing_char(str, n):
    return str[:n] + str[n+1:]
\end{verbatim}

\end{frame} 

% ---------------------------------------------
\begin{frame}[fragile]{Slicing}

{\Large you can skip items, too}

\begin{verbatim}
In [289]: string = "a fairly long string"

In [290]: string[0:15]
Out[290]: 'a fairly long s'

In [291]: string[0:15:2]
Out[291]: 'afil ogs'

In [292]: string[0:15:3]
Out[292]: 'aallg'
\end{verbatim}

\end{frame} 

% ---------------------------------------
\begin{frame}[fragile]{Command Line Input}

{\Large \verb|input| evaluates the input:}
\begin{verbatim}
In [265]: val = input("a message> ")
a message> 4.5
In [266]: type(val)
Out[266]: float
\end{verbatim}

{\Large \verb|raw_input| gives you the plain string:}
\begin{verbatim}
In [265]: val = input("a message> ")
a message> 4.5
In [266]: type(val)
Out[266]: float
\end{verbatim}

\vfill
(demo)
\end{frame}


%-------------------------------
\begin{frame}[fragile]{LAB}

\verb| def count_them(letter): |
\begin{itemize}
  \item prompts the user to input a letter
  \item counts the number of times the given letter is input
  \item prompts the user for another letter
  \item continues until the user inputs "x"
  \item returns the count of the letter input
\end{itemize}

\verb| def count_letter_in_string(string, letter): |
\begin{itemize}
  \item counts the number of instances of the letter in the string
  \item ends when a period is encountered
  \item if no period is encountered -- prints "hey, there was no period!"
\end{itemize}
\end{frame}

%-------------------------------
\begin{frame}[fragile]{LAB}
Write some functions that:
\begin{itemize}
  \item return a string with the first and last characters exchanged.
  \item return a string with every other character removed
  \item return a string with the first and last 4 characters removed, and every other char in between
  \item return a string reversed (just with slicing)
  \item return a string with the middle, then last, then first third in a new order
\end{itemize}

\end{frame}


%-------------------------------
\begin{frame}[fragile]{Homework}

Recommended Reading:
\begin{itemize}
  \item Read Think Python: 9, 14
  \item extra: string methods: \url{http://docs.python.org/library/stdtypes.html#string-methods}
  \item extra: unicode: \url{http://www.joelonsoftware.com/articles/Unicode.html}
\end{itemize}

Do:
\begin{itemize}
    \item Six more CodingBat exercises. 
    \item LPTHW: for extra practice with the concepts -- some of:
    \begin{description}
        \item[strings:] ex5, ex6, ex7, ex8, ex9, ex10
        \item[raw\_input(), sys.argv:] ex12, ex13, ex14 (needed for files)
        \item[files:] ex15, ex16, ex17 
    \end{description}    
\end{itemize}

\end{frame}

\end{document}

 
