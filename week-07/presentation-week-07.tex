\documentclass{beamer}
%\usepackage[latin1]{inputenc}
\usetheme{Warsaw}
\title[Intro to Python: Week 1]{Introduction  to Python\\ More OO -- Inheritance and Duck Typing \\
Special methods}

\author{Christopher Barker}
\institute{UW Continuing Education}
\date{November 12, 2013}

\usepackage{listings}
\usepackage{hyperref}

\begin{document}

% ---------------------------------------------
\begin{frame}
  \titlepage
\end{frame}

% ---------------------------------------------
\begin{frame}
\frametitle{Table of Contents}
%\tableofcontents[currentsection]
  \tableofcontents
\end{frame}


% ---------------------------------------------
\begin{frame}{Lightning Talks}

\vfill
{\LARGE Lightning talks today:}

\vfill
{\Large
 
\vfill
Linh Tran

\vfill
Maitri Kashyap  

\vfill
Sridharan Rajagopalan

\vfill
Richard Smith
}
\vfill

\end{frame}


\section{Review/Questions}

% ---------------------------------------------
\begin{frame}{Review of Previous Class}

{\LARGE
\begin{itemize}
  \item lambda
  \item Intro to OO
  \item Start of HTML generation code
\end{itemize}
}

\end{frame}


% ---------------------------------------------
\begin{frame}{Homework review}

  \vfill
  {\Large Questions? }

  \vfill
  {\Large Overview of my html-generating classes so far...}

\vfill
{\Large Demo of class vs. instance attributes}

\end{frame}



%-------------------------------
\begin{frame}{Lightning Talks}

{\LARGE Lightning Talks:}

\vfill
{\large Linh Tran}

\vfill
{\large Maitri Kashyap}

\end{frame}


%########################
\section{More on Subclassing}

% ---------------------------------------------
\begin{frame}[fragile]{Overriding \_\_init\_\_}

{\Large \verb|__init__|common method to override}
\vfill
{\large You often need to call the super class \verb|__init__| as well}
\vfill
\begin{verbatim}
class Circle(object):
    color = "red"
    def __init__(self, diameter):
        self.diameter = diameter
...
class CircleR(Circle):
    def __init__(self, radius):
        diameter = radius*2
        Circle.__init__(self, diameter)
\end{verbatim}
\vfill
exception to: "don't change the method signature" rule.
\end{frame} 

% ---------------------------------------------
\begin{frame}[fragile]{More subclassing}

{\large You can also call the superclass' other methods:}
\vfill
\begin{verbatim}
class Circle(object):
...
    def get_area(self, diameter):
        return math.pi * (diameter/2.0)**2

class CircleR2(Circle):
...
    def get_area(self):
        return Circle.get_area(self, self.radius*2)
\end{verbatim}

\vfill
There is nothing special about \verb|__init__| except that it gets called automatically.
\end{frame} 


% ---------------------------------------------
\begin{frame}[fragile]{When to Subclass}

\vfill
{\Large ``Is a'' relationship: Subclass/inheritance}

\vfill
{\Large ``Has a'' relationship: Composition}
\end{frame}

\begin{frame}[fragile]{When to Subclass}

{\Large ``Is a'' vs ``Has a'' }

\vfill
You may have a class that needs to accumulate an arbitrary number of objects.

\vfill
A list can do that -- so should you subclass list?

\vfill
Ask yourself:\\

\vfill
-- Is your class a list (with some extra functionality)?\\
\hspace{0.4in}or\\
-- Does you class HAVE a list?\\

\vfill
You only want to subclass list if your class could be used anywhere a list can be used.
\end{frame}

% ---------------------------------------------
\begin{frame}[fragile]{Attribute resolution order}

{\Large When you access an attribute:

\vfill
\hspace{0.2in}\verb|An_Instance.something|}

\vfill
{\Large Python looks for it in this order:}

\vfill
\begin{enumerate}
  \item Is it an instance attribute ?
  \item Is it a class attribute ?
  \item Is it a superclass attribute ?
  \item Is it a super-superclass attribute ?
  \item ...
\end{enumerate}

\vfill
It can get more complicated...\\
{\small
\url{http://www.python.org/getit/releases/2.3/mro/} \\
\url{http://python-history.blogspot.com/2010/06/method-resolution-order.html}
}
\end{frame} 

% ---------------------------------------------
\begin{frame}[fragile]{What are Python classes, really?}

{\Large Putting aside the OO theory...}

\vfill
{\Large Python classes are:}

\begin{itemize}
  \item Namespaces
  \begin{itemize}
    \item One for the class object
    \item One for each instance
  \end{itemize}
  \item Attribute resolution order
  \item Auto tacking-on of \verb|self|
\end{itemize}

\vfill
{\Large That's about it -- really!}

\end{frame} 

% ---------------------------------------------
\begin{frame}[fragile]{Type-Based dispatch}

{\Large From Think Python:}

\begin{verbatim}
  if isinstance(other, A_Class):
      Do_something_with_other
  else:
      Do_something_else
\end{verbatim}

\vfill
{\Large Usually better to use ``duck typing'' (polymorphism)}

\vfill
{\Large But when it's called for:}
\begin{itemize}
    \item \verb|isinstance()|
    \item \verb|issubclass()|
\end{itemize}

\vfill
GvR: ``Five Minute Multi- methods in Python'':\\
{\small \url{http://www.artima.com/weblogs/viewpost.jsp?thread=101605} }

\end{frame} 

%-------------------------------
\begin{frame}[fragile]{LAB}

{\Large We're going to do the rest: steps 4 - 8}

{(Still using \verb|week-06/code/htmlrender|) }

\vfill
{\Large Step 4:}

\begin{itemize}
  \item Extend the Element class to accept a set of attributes as keywords to the
  constructor, i.e.:
  \begin{verbatim}
Element("some text content",
        id="TheList",
        style="line-height:200\%")
  \end{verbatim}
    ( remember \verb|**kwargs| ? )
  \item The render method will need to be extended to render the attributes properly.
\end{itemize}

\vfill
You can now render some \verb|<p>| tags (and others) with attributes  
\end{frame}

%-------------------------------
\begin{frame}[fragile]{LAB}

{\Large Step 5:}

\begin{itemize}
   \item Create a \verb|SelfClosingTag| subclass of \verb|Element|, to render tags like:
   
   \verb|<hr /> and <br />| (horizontal rule and line break).
   
   \item You will need to override the render method to render just the one tag and
   attributes.
   
   \item create a couple subclasses of SelfClosingTag for \verb|<hr>|
   and \verb|<br />| (Line break) or ??? if you like
   \end{itemize}

\vfill
You can now render an html page with a proper \verb|<head>| (\verb|<meta />| and \verb|<title>| elements)
\end{frame}

\begin{frame}[fragile]{LAB}

{\Large Step 6:}

\begin{itemize}
   \item  Create an \verb|A| class for an anchor (link) element. Its constructor should
          look like: \verb|A(self, link, content)| -- where link is the link,
          and content is what you see. It can be called like so:

   \verb|A("http://google.com", "link")|

  \item You should be able to subclass from \verb|Element|, and only override
        the \verb|__init__|\\
        -- Calling the \verb|Element __init__| from the  \verb|A __init__|
\end{itemize}

\vfill
    You can now add a link to your web page.
\end{frame}

\begin{frame}[fragile]{LAB}

{\Large Step 7:}

\begin{itemize}
   \item Create \verb|Ul| class for an unordered list (really simple subclass of Element)
   
   \item Create \verb|Li| class for an element in a list (also really simple)
   
   \item add a list to your web page.
   
   \item Create a Header class -- this one should take an integer argument for the
   header level. i.e \verb|<h1>, <h2>, <h3>|, called like:
   
   \item \verb|H(2, "The text of the header")| for an \verb|<h2>| header
   
   \item It can subclass from \verb|OneLineTag| -- overriding the \verb|__init__|, then calling
       the superclass \verb|__init__|
\end{itemize}

\end{frame}

\begin{frame}[fragile]{LAB}

{\Large Step 8:}

\begin{itemize}
   \item Update the Html element class to render the "\verb|<!DOCTYPE html>|" tag at the
   head of the page, before the \verb|html| element.
   
   \item You can do this by subclassing \verb|Element|, overriding \verb|render()|, but then
   calling \verb|Element.render()| from \verb|Html.render()|.

   \item Create a subclass of \verb|SelfClosingTag| for \verb|<meta charset="UTF-8" />|
   and add the meta element to the beginning of the head element to give your document
   an encoding.
   
   \item The doctype and encoding are HTML 5 and you can check this at:
          \url{validator.w3.org.}

\end{itemize}

\vfill
You now have a pretty full-featured html renderer
\end{frame}

% ---------------------------------------------
\begin{frame}[fragile]{Review of HTML renderer lab}

{\Large You have built an html generator, using:} 
  \begin{itemize}
    \item A Base Class with a couple methods
    \item Subclasses overriding class attributes
    \item Subclasses overriding a method
    \item Subclasses overriding the \verb|__init__|
  \end{itemize}

\vfill
{\Large These are the core OO approaches}

\vfill
{\Large If you don't have it working, or don't think you ``get'' it:\\
 work on it for homework, and ask questions.}

\end{frame}


%-------------------------------
\begin{frame}{Lightning Talks}

{\LARGE Lightning Talks:}

\vfill
{\large Sridharan Rajagopalan}

\vfill
{\large Richard Smith}

\end{frame}

%%%%%%%%%%%%%%%%%%%%%%%%%%%%%%%
\section{Multiple Inheritance}

% ---------------------------------------------
\begin{frame}[fragile]{multiple inheritance}

{\Large Multiple inheritance:\\
\hspace{0.2in} Pulling from more than one class}

\vfill
\begin{verbatim}
class Combined(Super1, Super2, Super3):
    def __init__(self, something, something else):
        Super1.__init__(self, ......)        
        Super2.__init__(self, ......)        
        Super3.__init__(self, ......)        
\end{verbatim}
(calls to the super class \verb|__init__| are optional -- case dependent)

\end{frame} 

% ---------------------------------------------
\begin{frame}[fragile]{multiple inheritance}

\vfill
{\Large Attribute resolution -- left to right}

\begin{enumerate}
  \item Is it an instance attribute ?
  \item Is it a class attribute ?
  \item Is it a superclass attribute ?
  \begin{enumerate}
     \item is the it an attribute of the left-most superclass?
     \item is the it an attribute of the next superclass?
     \item ....
  \end{enumerate}
  \item Is it a super-superclass attribute ?
  \item ...also left to right...
\end{enumerate}

\vfill
\url{http://python-history.blogspot.com/2010/06/method-resolution-order.html}
\end{frame} 

% ---------------------------------------------
\begin{frame}[fragile]{Mix-ins}

{\Large Why would you want to do this?}

\vfill
{\Large Hierarchies are not always simple:}
\vfill
{\large
\begin{itemize}
  \item Animal
  \begin{itemize}
    \item Mammal
    \begin{itemize}
      \item GiveBirth()
    \end{itemize}
    \item Bird
    \begin{itemize}
      \item LayEggs()
    \end{itemize}
  \end{itemize}
\end{itemize}
}

\vfill
{\Large Where do you put a Platypus or an Armadillo?}

\vfill
{\Large Real World Example: \verb|FloatCanvas|}
\end{frame} 


%-------------------------------
\begin{frame}[fragile]{New Style classes}

{\Large You will see reference to ``new style'' classes}

\vfill
{\Large These derive from \verb|object|}

\vfill
{\Large Introduced in python2.2 to better merge types and classes, and clean up a few things}

\vfill
{\Large Differences in method resolution order and properties}

\vfill
{\Large Mostly the same, often makes no difference}

\vfill
{\Large My advice: always subclass from \verb|object|}

\end{frame}

%-----------------------------------
\begin{frame}[fragile]{super}

{\Large \verb|super(): |use it to call a superclass method, rather than exlicitly calling it:}

\vfill
{\large instead of:}
\begin{verbatim}
class A(B):
    def __init__(self, *args, **kwargs)
        B.__init__(self, *argw, **kwargs)
        ...
\end{verbatim}

{\large You can do:}
\begin{verbatim}
class A(B):
    def __init__(self, *args, **kwargs)
        super(A, self).__init__(self, *argw, **kwargs)
        ...
\end{verbatim}

\vfill
{\Large There are some subtle differences with multiple inheritance}

\end{frame}

%--------------------------
\begin{frame}[fragile]{super}

{\Large Two seminal articles about \verb|super()|:}

\vfill
{\LARGE ``Super Considered Harmful''}\\[0.1in]
{\Large \hspace{0.5in}-- James Knight }

\vfill
\url{https://fuhm.net/super-harmful/}

\vfill
{\LARGE ``super() considered super!''}\\[0.1in]
{\Large  \hspace{0.5in}--  Raymond Hettinger }

\vfill
\url{http://rhettinger.wordpress.com/2011/05/26/super-considered-super/}
\vfill

{\large (Both worth reading....)}
\end{frame} 




%-------------------------------
\begin{frame}[fragile]{Wrap Up}

{\LARGE Thinking OO in Python:}

\vfill
{\large Think about what makes sense for your code:}
\begin{itemize}
  \item {\large Code re-use}
  \item {\large Clean APIs}
  \item {\large ... }
\end{itemize}

\vfill
{\large Don't be a slave to what OO is \emph{supposed} to look like. }

\vfill
{\large Let OO work for you, not \emph{create} work for you}

\end{frame}


%-------------------------------
\begin{frame}[fragile]{Wrap Up}

{\Large OO in Python:}

\vfill
{\Large The Art of Subclassing}: Raymond Hettinger

\vfill
{\small \url{http://pyvideo.org/video/879/the-art-of-subclassing}}

\vfill
''classes are for code re-use -- not creating taxonomies''

\vfill
{\Large Stop Writing Classes}: Jack Diederich

\vfill
{\small \url{http://pyvideo.org/video/880/stop-writing-classes}}

\vfill
``If your class has only two methods -- and one of them is \verb|__init__|
-- you don't need a class ''
\end{frame}

%-------------------------------
\begin{frame}[fragile]{Homework}

{\Large Finish the labs.}

\vfill
{\Large Watch the videos.}

\vfill
{\Large Readup more on OO design.}


\vfill
{\LARGE Your Project:}
\begin{itemize}
  \item By next week, send me a project proposal: can be short and sweet.
  \item Think about how you might use OO:
  \begin{itemize}
    \item What classes naturally fall out of the problem?
    \item NOTE: maybe none!
  \end{itemize}
\end{itemize}

\end{frame}
 

\end{document}

 
