\documentclass{beamer}
%\usepackage[latin1]{inputenc}
\usetheme{Warsaw}
\title[Intro to Python: Week 1]{Introduction  to Python\\
Unicode, Advanced Argument passing\\
List and Dict Comprehensions, Testing
}
\author{Christopher Barker}
\institute{UW Continuing Education}
\date{October 29, 2013}

\usepackage{listings}
\usepackage{hyperref}

\begin{document}

% ---------------------------------------------
\begin{frame}
  \titlepage
\end{frame}

% ---------------------------------------------
\begin{frame}
\frametitle{Table of Contents}
%\tableofcontents[currentsection]
  \tableofcontents
\end{frame}


\section{Review/Questions}

% ---------------------------------------------
\begin{frame}{Review of Previous Class}

\begin{itemize}
  \item Dictionaries
  \item Exceptions
  \item Files, etc.
\end{itemize}

\end{frame}


% ---------------------------------------------
\begin{frame}{Lightning Talks}

\vfill
{\LARGE Lightning talks today:}

\vfill
{\Large
Rithy Chhen

\vfill
Howard Edson

\vfill
Dong Kang

\vfill
Steven Werner

}
\vfill

\end{frame}


% ---------------------------------------------
\begin{frame}[fragile]{Homework review}

  \vfill
  {\Large Homework Questions? }

  \vfill
  {\Large My Solution}

  (\verb|dict.setdefault()| trick...)

  \vfill
\end{frame}

%%%%%%%%%%%%%%%%%%%%%%%%%%%%%%%%%
\section{Unicode}

%---------------------------------
\begin{frame}[fragile]{Unicode}

{\Large I hope you all read this:}

\vfill
{\Large
\centering
The Absolute Minimum Every Software Developer Absolutely,
Positively Must Know About Unicode and Character Sets (No Excuses!)

}

\vfill
\url{http://www.joelonsoftware.com/articles/Unicode.html}

\vfill
{\Large If not -- go read it!}

\end{frame}

\begin{frame}[fragile]{Unicode}

{\Large
\vfill

Everything is bytes

\vfill
If it's on disk or transmitted over a network, it's bytes

\vfill
Python provides some abstractions to make it easier to deal with bytes

\vfill
}

\end{frame}

\begin{frame}[fragile]{Unicode}

{\Large
\vfill

Unicode is a biggie

\vfill
Strings vs Unicode 
}

{\large (\verb|str()| vs. \verb|bytes()| vs. \verb|unicode()| ) }

\vfill
{\Large Python 2.x vs 3.x}


\vfill
(actually, dealing with numbers rather than bytes is big -- but we take that for granted)

\end{frame}

\begin{frame}[fragile]{Unicode}

{\Large
\vfill
Strings are sequences of bytes

\vfill
Unicode strings are sequences of platonic characters

\vfill
Platonic characters cannot be written to disk or network!
}
\vfill
(ANSI -- one character == one byte -- so easy!)
\end{frame}

\begin{frame}[fragile]{Unicode}

{\Large
\vfill
The \verb|unicode| object lets you work with characters

\vfill
``Encoding'' is converting from a unicode object to bytes

\vfill
``Decoding'' is converting from bytes to a unicode object
}

\vfill
\end{frame}

\begin{frame}[fragile]{Unicode}

{\large
\begin{verbatim}
import codecs
ord()
chr()
unichr()
str()
unicode()
encode()
decode()
\end{verbatim}
}
\end{frame}

\begin{frame}[fragile]{Unicode Literals}


{\Large 1) Use unicode in your source files:}

\begin{verbatim}
# -*- coding: utf-8 -*-
\end{verbatim}

\vfill
{\Large 2) escape the unicode characters}

\begin{verbatim}
print u"The integral sign: \u222B"
print u"The integral sign: \N{integral}"
\end{verbatim}

{\large lots of tables of code points online:}

\url{http://inamidst.com/stuff/unidata/}

\vfill
(demo: \verb|code\hello_unicode.py|)
\end{frame}


%---------------------------------
\begin{frame}[fragile]{Unicode}

{\Large
Use unicode objects in all your code

\vfill
decode on input

\vfill
encode on output

\vfill
Many packages do this for you\\
\hspace{0.25in} (XML processing, databases, ...)

\vfill
Gotcha:\\
\hspace{0.25in} Python has a default encoding (usually ascii)
}
\end{frame}

\begin{frame}[fragile]{Unicode}

{\Large Python Docs Unicode HowTo:}

\url{http://docs.python.org/howto/unicode.html}

\vfill
{\Large ``Reading Unicode from a file is therefore simple:''}

\begin{verbatim}
import codecs
f = codecs.open('unicode.rst', encoding='utf-8')
for line in f:
    print repr(line)
\end{verbatim}

\vfill
{\Large Encodings Built-in to Python:}

\url{http://docs.python.org/2/library/codecs.html#standard-encodings}

\end{frame}

%-------------------------------
\begin{frame}[fragile]{Unicode LAB}

\begin{itemize}
  \item Find some nifty non-ascii characters you might use.\\
        Create a unicode object with them in two different ways.
  \item In the "code" dir for this week, there are two files:\\
        \verb|text.utf16| \\
        \verb|text.utf32| \\
        read the contents into unicode objects
  \item write some of the text from the first exercise to file.
  \item read that file back in.
\end{itemize}

\vfill
(reference: \url{http://inamidst.com/stuff/unidata/})

\vfill
NOTE: if your terminal does not support unicode -- you'll get an error trying
to print. Try a different terminal or IDE, or google for a solution
\end{frame}



%-------------------------------
\begin{frame}{Lightning Talk}

{\LARGE Lightning Talks:}

{\large 
\vfill
Rithy Chhen

\vfill
Howard Edson
}

\end{frame}

%%%%%%%%%%%%%%%%%%%%%%%%%%%%%%%%%%%%%%%%%%%%
\section{Advanced Argument Passing}


% ---------------------------------------------
\begin{frame}[fragile]{Keyword arguments}

 {\Large When defining a function, you can specify only
         what you need -- any order}

\begin{verbatim}
In [151]: def fun(x,y=0,z=0):
        print x,y,z
   .....:     

In [152]: fun(1,2,3)
1 2 3

In [153]: fun(1, z=3)
1 0 3

In [154]: fun(1, z=3, y=2)
1 2 3
\end{verbatim}

\end{frame} 

% ---------------------------------------------
\begin{frame}[fragile]{Keyword arguments}

 {\Large A Common Idiom:}

\vfill
\begin{verbatim}
def fun(x, y=None):
    if y is None:
        do_something_different

    go_on_here
\end{verbatim}
\vfill

\end{frame} 

% ---------------------------------------------
\begin{frame}[fragile]{Keyword arguments}

 {\Large Can set defaults to variables}

\begin{verbatim}
In [156]: y = 4

In [157]: def fun(x=y):
    print "x is:", x
   .....:     

In [158]: fun()
x is: 4

\end{verbatim}

\end{frame} 

% ---------------------------------------------
\begin{frame}[fragile]{Keyword arguments}

{\Large Defaults are evaluated when the function is defined}

\begin{verbatim}
In [156]: y = 4

In [157]: def fun(x=y):
    print "x is:", x
   .....:     

In [158]: fun()
x is: 4

In [159]: y = 6

In [160]: fun()
x is: 4
\end{verbatim}

\end{frame} 


% ---------------------------------------------
\begin{frame}[fragile]{Function arguments in variables}

{\Large function arguments are really just\\
 -- a tuple (positional arguments) \\
 -- a dict (keyword arguments) \\
}
\begin{verbatim}
def f(x, y, w=0, h=0):
    print "position: %s, %s -- shape: %s, %s"%(x, y, w, h)

position = (3,4)
size = {'h': 10, 'w': 20}

>>> f( *position, **size)
position: 3, 4 -- shape: 20, 10
\end{verbatim}

\end{frame} 

% ---------------------------------------------
\begin{frame}[fragile]{Function parameters in variables}

{\Large You can also pull in the parameters out in the function as a tuple and a dict
}
\begin{verbatim}
def f(*args, **kwargs):
    print "the positional arguments are:", args
    print "the keyword arguments are:", kwargs
 
In [389]: f(2, 3, this=5, that=7)
the positional arguments are: (2, 3)
the keyword arguments are: {'this': 5, 'that': 7}
\end{verbatim}

\end{frame} 

%-------------------------------
\begin{frame}[fragile]{LAB}

{\Large keyword arguments}
\begin{itemize}
  \item Write a function that has four optional parameters\\
        (with defaults):
  \begin{itemize}
      \item foreground\_color
      \item background\_color
      \item link\_color
      \item visited\_link\_color
  \end{itemize}
  \item Have it print the colors.
  \item Call it with a couple different parameters set
  \item Have it pull the parameters out with \verb|*args, **kwargs| 
\end{itemize}

\end{frame}

%%%%%%%%%%%%%%%%%%%%%%%%%%%%%%%%%%%%%%
\section{List and Dict Comprehensions}

% ----------------------------------------------
\begin{frame}[fragile]{List comprehensions}

{\Large A bit of functional programming:}

\begin{verbatim}
new_list = [expression for variable in a_list]
\end{verbatim}

{\Large same as for loop:}

\begin{verbatim}
new_list = []
for variable in a_list:
    new_list.append(expression)
\end{verbatim}

\end{frame} 

% ----------------------------------------------
\begin{frame}[fragile]{List comprehensions}

{\Large More than one ``for'':}

\begin{verbatim}
new_list = \
[exp for var in a_list for var2 in a_list2]
\end{verbatim}

{\Large same as nested for loop:}

\begin{verbatim}
new_list = []
for var in a_list:
    for var2 in a_list2:
        new_list.append(expression)
\end{verbatim}

{\large You get the ``outer product'', i.e. all combinations.}

\vfill
(demo)
\end{frame} 

% ----------------------------------------------
\begin{frame}[fragile]{List comprehensions}

{\Large Add a conditional:}

\begin{verbatim}
new_list = \
[expression for variable in a_list if something_is_true]
\end{verbatim}

{\Large same as for loop:}

\begin{verbatim}
new_list = []
for variable in a_list:
    if something_is_true:
        new_list.append(expression)
\end{verbatim}

\vfill
(demo)
\end{frame} 



% ----------------------------------------------
\begin{frame}[fragile]{List comprehensions}

{\Large Examples:}

\begin{verbatim}
In [341]: [x**2 for x in range(3)]
Out[341]: [0, 1, 4]

In [342]: [x+y for x in range(3) for y in range(5,7)]
Out[342]: [5, 6, 6, 7, 7, 8]

In [343]: [x*2 for x in range(6) if not x%2]
Out[343]: [0, 4, 8]
\end{verbatim}

\end{frame} 


% ----------------------------------------------
\begin{frame}[fragile]{List comprehensions}

{\Large Remember this from last week?}

\begin{verbatim}
[name for name in dir(__builtin__) if "Error" in name]

['ArithmeticError',
 'AssertionError',
 'AttributeError',
 'BufferError',
 'EOFError',
 ....
\end{verbatim}

\end{frame} 

% ----------------------------------------------
\begin{frame}[fragile]{Set Comprehensions}

{\Large You can do it with sets, too:}

\begin{verbatim}
new_set = { value for variable in a_sequence}
\end{verbatim}

{\Large same as for loop:}

\begin{verbatim}
new_set = set()
for key in a_list:
    new_set.add(value)
\end{verbatim}

\end{frame} 

% ----------------------------------------------
\begin{frame}[fragile]{Set Comprehensions}


\begin{verbatim}
In [33]: s = "a fairly long string"

In [34]: vowels = 'aeiou'

In [35]: { l for l in s if l in vowels}
Out[35]: set(['a', 'i', 'o'])
\end{verbatim}

\end{frame} 



% ----------------------------------------------
\begin{frame}[fragile]{Dict Comprehensions}

{\Large and with dicts:}

\begin{verbatim}
new_dict = { key:value for variable in a_sequence}
\end{verbatim}

{\Large same as for loop:}

\begin{verbatim}
new_dict = {}
for key in a_list:
    new_dict[key] = value
\end{verbatim}

\end{frame} 

% ----------------------------------------------
\begin{frame}[fragile]{Dict Comprehensions}

{\Large Example}

\begin{verbatim}
In [340]: { i: "this_%i"%i for i in range(5) }
Out[340]: {0: 'this_0', 1: 'this_1', 2: 'this_2',
           3: 'this_3', 4: 'this_4'}
\end{verbatim}

\vfill
(not as useful with the \verb|dict()| constructor...)
\end{frame} 


%-------------------------------
\begin{frame}[fragile]{LAB}

\vfill
{\Large List and Dict comprehension lab:}

\vfill
{\large \verb|code/comprehensions.rst[html]| }

\vfill

\end{frame}


%-------------------------------
\begin{frame}{Lightning Talk}

{\LARGE Lightning Talks:}

{\large 
\vfill
Dong Kang

\vfill
Steven Werner
}
\vfill
\end{frame}


\section{Unit Testing}

% ---------------------------------------------
\begin{frame}[fragile]{Unit Testing}

{\LARGE Gaining Traction}

\vfill
{\Large You need to test your code somehow when you write it --
        why not preserve those tests?}

\vfill
{\Large And allow you to auto-run them later?}

\vfill
{\LARGE Test-Driven development:}\\[0.1in]
{\Large \hspace{0.3in} Write the tests before the code}

\end{frame} 

% ---------------------------------------------
\begin{frame}[fragile]{Unit Testing}

{\LARGE My thoughts:}

\vfill
{\Large Unit testing encourages clean, decoupled design}

\vfill
{\Large If it's hard to write unit tests for -- it's not well designed}

\vfill
{\Large but...}

\vfill
{\Large ``complete'' test coverage is a fantasy}

\end{frame} 


% ---------------------------------------------
\begin{frame}[fragile]{PyUnit}

{\LARGE PyUnit: the stdlib unit testing framework}

\vfill
{\Large \verb|import unittest|}

\vfill
{\Large More or less a port of Junit from Java}

\vfill
{\Large A bit verbose: you have to write classes \& methods}

\vfill
{\large (And we haven't covered that yet!)}

\end{frame} 

% ---------------------------------------------
\begin{frame}[fragile]{unittest example}

{\small
\begin{verbatim}
import random
import unittest

class TestSequenceFunctions(unittest.TestCase):

    def setUp(self):
        self.seq = range(10)

    def test_shuffle(self):
        # make sure the shuffled sequence does not lose any elements
        random.shuffle(self.seq)
        self.seq.sort()
        self.assertEqual(self.seq, range(10))

        # should raise an exception for an immutable sequence
        self.assertRaises(TypeError, random.shuffle, (1,2,3))
\end{verbatim}
}

\end{frame} 


% ---------------------------------------------
\begin{frame}[fragile]{unittest example (cont)}

{\small
\begin{verbatim}
    def test_choice(self):
        element = random.choice(self.seq)
        self.assertTrue(element in self.seq)

    def test_sample(self):
        with self.assertRaises(ValueError):
            random.sample(self.seq, 20)
        for element in random.sample(self.seq, 5):
            self.assertTrue(element in self.seq)

if __name__ == '__main__':
    unittest.main()
\end{verbatim}
}

\vfill
(\verb|code/unitest_example.py|)

\vfill
\url{http://docs.python.org/library/unittest.html}

\end{frame} 

% ---------------------------------------------
\begin{frame}[fragile]{unittest}

{\Large Lots of good tutorials out there:}

\vfill
{\Large Google: ``python unittest tutorial''}

\vfill
{\Large I first learned from this one:}\\[0.1in]
\url{http://www.diveintopython.net/unit_testing/index.html}

\end{frame} 

% ---------------------------------------------
\begin{frame}[fragile]{nose and pytest}

{\Large Due to its Java heritage, unittest is kind of verbose}

\vfill
{\Large Also no test discovery}\\
{\large \hspace{0.2in}(though unittest2 does add that...) }

\vfill
{\Large So folks invented nose and pytest}

\end{frame} 

\begin{frame}[fragile]{nose}

{\LARGE \verb|nose|}

\vfill
{\Large \hspace{0.2in} Is nicer testing for python}

\vfill
{\Large \hspace{0.2in} nose extends unittest to make testing easier.}

\vfill
\begin{verbatim}
 $ pip install nose

 $ nosetests unittest_example.py 
\end{verbatim}

\vfill
\url{http://nose.readthedocs.org/en/latest/}
\end{frame} 

\begin{frame}[fragile]{nose example}

{\Large The same example -- with nose}

{\small
\begin{verbatim}
import random
import nose.tools

seq = range(10)

def test_shuffle():
    # make sure the shuffled sequence does not lose any elements
    random.shuffle(seq)
    seq.sort()
    assert seq == range(10)

@nose.tools.raises(TypeError)
def test_shuffle_immutable():
    # should raise an exception for an immutable sequence
    random.shuffle( (1,2,3) )
\end{verbatim}
}

\end{frame} 

\begin{frame}[fragile]{nose example (cont) }

{\small
\begin{verbatim}
def test_choice():
    element = random.choice(seq)
    assert (element in seq)

def test_sample():
    for element in random.sample(seq, 5):
        assert element in seq

@nose.tools.raises(ValueError)
def test_sample_too_large():
    random.sample(seq, 20)
\end{verbatim}
}

\vfill
(\verb|code/test_random_nose.py|)

\end{frame} 


\begin{frame}[fragile]{pytest}

{\LARGE \verb|pytest|}

\vfill
{\Large \hspace{0.2in} A mature full-featured testing tool}

\vfill
{\Large \hspace{0.2in} Provides no-boilerplate testing}

\vfill
{\Large \hspace{0.2in} Integrates many common testing methods}

\vfill
\begin{verbatim}
 $ pip install pytest

 $ py.test unittest_example.py 
\end{verbatim}

\vfill
\url{http://pytest.org/latest/}
\end{frame} 

\begin{frame}[fragile]{pytest example}

{\Large The same example -- with pytest}

{\small
\begin{verbatim}
import random
import pytest

seq = range(10)

def test_shuffle():
    # make sure the shuffled sequence does not lose any elements
    random.shuffle(seq)
    seq.sort()
    assert seq == range(10)

def test_shuffle_immutable():
    pytest.raises(TypeError, random.shuffle,  (1,2,3) )
\end{verbatim}
}

\end{frame} 

\begin{frame}[fragile]{pytest example (cont) }

{\small
\begin{verbatim}
def test_choice():
    element = random.choice(seq)
    assert (element in seq)

def test_sample():
    for element in random.sample(seq, 5):
        assert element in seq

def test_sample_too_large():
    with pytest.raises(ValueError):
        random.sample(seq, 20)
\end{verbatim}
}

\vfill
(\verb|code/test_random_pytest.py|)

\end{frame} 


\begin{frame}[fragile]{Parameterized Tests}

{\Large A whole set of inputs and outputs to test?}

\vfill
{\Large \verb|pytest| has a nice way to do that (so does nose...)}

\begin{verbatim}
import pytest
@pytest.mark.parametrize(("input", "expected"), [
    ("3+5", 8),
    ("2+4", 6),
    ("6*9", 42),
])
def test_eval(input, expected):
    assert eval(input) == expected
\end{verbatim}

\url{http://pytest.org/latest/example/parametrize.html}

\vfill
(\verb|code/test_pytest_parameter.py|)
\end{frame} 

\begin{frame}[fragile]{Test Coverage}

{\LARGE \verb|coverage.py |}

\vfill
{\Large Uses debugging hook to see which lines of code are actually executed
-- plugins exist for most (all?) test runners}

\vfill
{\Large \verb|pip install coverage |}

\vfill
{\Large \verb|nosetests --with-coverage test_codingbat.py|}

\vfill
\url{http://nedbatchelder.com/code/coverage/}
\end{frame}

% --------------------------
\begin{frame}[fragile]{Coding Bat}

{\LARGE Coding Bat:}

\url{http://codingbat.com/python}


\vfill
{\Large Tells you what unit tests to write:}

\url{http://codingbat.com/prob/p118406}

\vfill
{\Large We'll use them for our lab}

\end{frame}

%-------------------------------
\begin{frame}[fragile]{LAB}

{\Large First: get pip installed:}

{\small \url{http://www.pip-installer.org/en/latest/installing.html} }

\vfill
{\Large Second: install nose and/or pytest:}

{\large \verb|pip install nose| -- \verb|pip install pytest|}


\vfill
{\Large Unit Testing:}

\begin{itemize}
  % \item unittest
  %   \begin{itemize}
  %      \item Pick a \url{codingbat.com} example
  %      \item Write a set of unit tests using \verb|unittest|
  %        (\verb|code\codingbat.py  codingbat_unittest.py|)
  %   \end{itemize}
  \item pytest / nose
    \begin{itemize}
       \item Test a \url{codingbat.com} with nose or pytest
       \item Try doing test-driven development
         (\verb|code\test_codingbat.py|)
    \end{itemize}

  \item try running \verb|coverage| on your tests
\end{itemize}

\end{frame}



%-------------------------------
\begin{frame}[fragile]{Homework}

Recommended Reading:
\begin{itemize}
  \item TP: ch 15-18
  \item LPTHW: Ex 40 - 45
  \item Dive Into Python: chapter 4, 5 
\end{itemize}

Do:
\begin{itemize}
  \item Finish (or re-factor) the Labs you didn't finish in class.
  \item Write some unit tests for a couple of the functions you've
        written for previous excercises (Or something new)
  \item Using the unit tests you jsut wrote, refactor the above functions
        using list and/or dict comprehensions.
  \item Write a script which does something useful (to you) and reads and writes
        files. Very, very small scope is good. something useful at work would
        be great, but no job secrets!
  \item Start thinking about what you want to do for your project!
\end{itemize}


\end{frame}


\end{document}

 
