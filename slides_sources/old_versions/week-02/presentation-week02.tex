\documentclass{beamer}
%\usepackage[latin1]{inputenc}
\usetheme{Warsaw}
\title[Intro to Python: Week 2]{Introduction  to Python\\ Functions, Booleans, Modules}
\author{Christopher Barker}
\institute{UW Continuing Education}
\date{October 8, 2013}
%-------------------------------

\usepackage{listings}
\usepackage{hyperref}

\begin{document}

% ---------------------------------------------
\begin{frame}
  \titlepage
\end{frame}

% ---------------------------------------------
\begin{frame}
\frametitle{Table of Contents}
%\tableofcontents[currentsection]
  \tableofcontents
\end{frame}


\section{Review/Questions}

% ---------------------------------------------
\begin{frame}[fragile]{Review of Previous Class}

\begin{itemize}
  \item Values and Types
  \item Expressions
  \item Intro to functions
\end{itemize}

\end{frame}


% ---------------------------------------------
\begin{frame}[fragile]{Lightning Talks}

\vfill
{\LARGE Lightning talks today:}

\vfill
{\Large
Jo-Anne Antoun 

\vfill
Omer Onen 

\vfill
Ryan Small

\vfill
Catherine Warren 
}
\vfill

\end{frame}


% ---------------------------------------------
\begin{frame}[fragile]{Homework review}

  \vfill
  {\Large Homework Questions? }

  \vfill
  {\Large To loop or not to loop?}

  \vfill
  {\Large Build up strings, then print...}

  \vfill
  {\Large My Solution}

  \vfill

\end{frame}

% ---------------------------------------------
\begin{frame}[fragile]{Stuff brought up by homework}

  \vfill
  {\Large Bytecode and \verb|*.pyc| }

  \vfill
  {\Large Please send me code:
    \begin{itemize}
      \item Enclosed in an email
      \item With your name at the beginning of the filename: \verb|chris_problem1.py|
    \end{itemize}
  }

  \vfill
  {\Large PEP 8}

  \vfill
  {\Large Repeating variable names in nested loops}


\end{frame}


\section{Quick Intro to Basics}

\begin{frame}[fragile]{Basics}

\vfill
{\LARGE It turns out you can't really do much at all without at least a container type, conditionals and looping...}
\vfill

\end{frame}

%-------------------------------
\begin{frame}[fragile]{if}

{\Large Making Decisions...}
\begin{verbatim}
if a:
    print 'a'
elif b:
    print 'b'
elif c:
    print 'c'
else:
    print 'that was unexpected'
\end{verbatim}

\end{frame}


%-------------------------------
\begin{frame}[fragile]{if}

{\Large Making Decisions...}
\begin{verbatim}
if a:
    print 'a'
elif b:
    print 'b'

## versus...

if a:
    print 'a'
if b:
    print 'b'
\end{verbatim}

\end{frame}

%-------------------------------
\begin{frame}[fragile]{switch?}

\vfill
{\Large No switch/case in Python}

\vfill
{\Large use \verb|if..elif..elif..else|}

\vfill

(or a dictionary, or subclassing....)
\end{frame}

%-------------------------------
\begin{frame}[fragile]{lists}

\vfill
{\Large A way to store a bunch of stuff in order}

\vfill
{\large ``array'' in other languages}

\vfill
\begin{verbatim}
a_list = [2,3,5,9]

a_list_of_strings = ['this', 'that', 'the', 'other']
\end{verbatim}

\vfill

\end{frame}

%-------------------------------
\begin{frame}[fragile]{tuples}

\vfill
{\Large Another way to store an ordered list of things}

\vfill
\begin{verbatim}
a_tuple = (2,3,4,5)

a_tuple_of_strings = ('this', 'that', 'the', 'other')
\end{verbatim}

\vfill
{\Large Often interchangeable with lists, but not always...}

\end{frame}


%-------------------------------
\begin{frame}[fragile]{for}

{\Large When you need to do something to everything in a sequence}

\vfill
\begin{verbatim}
>> a_list = [2,3,5,9]

>> for item in a_list:
>>     print item
2
3
5
9
\end{verbatim}

\end{frame}

%-------------------------------
\begin{frame}[fragile]{range() and for}

{\Large When you need to do something a set number of times}

\vfill
\begin{verbatim}
>>> range(4)
[0, 1, 2, 3]
>>> for i in range(6):
...     print "*",
... 
* * * * * *
>>> 
\end{verbatim}

\end{frame}

%-------------------------------
\begin{frame}[fragile]{intricacies}

\vfill
{\Large This is enough to get you started.}

\vfill
{\Large Each of these have intricacies special to python}

\vfill
{\Large We'll get to those over the next couple classes}

\vfill

\end{frame}

%%%%%%%%%%%%%%%%%%%%%%%%%%%%%
\section{More on Functions}

%-------------------------------
\begin{frame}[fragile]{Functions: review}

{\Large Defining a function:}

\begin{verbatim}
def fun(x, y):
    z = x+y
    return z
\end{verbatim}

{\Large x, y, z are local names}

\end{frame}


\begin{frame}[fragile]{Functions: local vs. global}

\begin{verbatim}
x = 32
def fun(y, z):
    print x, y, z

fun(3,4)

32 3 4
\end{verbatim}
{\large x is global, y and z local}

\vfill
{\Large Use global variables mostly for constants}

\end{frame}


%----------------------------------
\begin{frame}[fragile]{Recursion}

\vfill
{\LargeRecursion is calling a function from itself.}

\vfill
{\LargeMax stack depth, function call overhead.}

\vfill
{\LargeBecause of these two(?), recursion isn't used {\bf that} often in Python.}

\vfill
(demo: factorial)
\end{frame}

%----------------------------------
\begin{frame}[fragile]{Tuple Unpacking}

{\Large Remember: \verb| x,y = 3,4| ?}

\vfill
{\Large Really ``tuple unpacking'': \verb| (x, y) = (3, 4)|}

\vfill
{\Large This works in function arguments, too:}

\begin{verbatim}
>>> def a_fun( (a, b), (c, d) ):
...     print a, b, c, d
... 
>>> t, u = (3,4), (5,6)
>>> 
>>> a_fun(t, u)
3 4 5 6
\end{verbatim}
(demo)
\end{frame}


%----------------------------------
\begin{frame}[fragile]{Lab: more with functions}

{\Large Write a function that:}
\begin{itemize}
  \item computes the distance between two points:\\
        dist = sqrt( (x1-x2)**2 + (y1-y2)**2 )\\
        using tuple unpacking...
  \item Take some code with functions, add this to each function:\\
        \verb|print locals()|
  \item Computes the Fibonacci series with a recursive function:\\
  f(0) = 0; f(1) = 1\\
  f(n) = f(n-1) + f(n-2)\\
  0, 1, 1, 2, 3, 5, 8, 13, 21, ...\\
  (If time: a non-recursive version)
\end{itemize}

\end{frame}

%-------------------------------
\begin{frame}[fragile]{Lightning Talks}

\vfill
{\LARGE Lightning Talks:}

\vfill
{\Large Jo-Anne Antoun }

\vfill
{\Large Omer Onen }

\vfill
\end{frame}



%%%%%%%%%%%%%%%%%%%%%%%%%%%%%%%%%
\section{Boolean Expressions}

% ---------------------------------------------
\begin{frame}[fragile]{Truthiness}

{\Large What is true or false in Python?}

\begin{itemize}
  \item The Booleans: \verb+True+ and \verb+False+
  \item ``Something or Nothing''
\end{itemize}

{\small \url{http://mail.python.org/pipermail/python-dev/2002-April/022107.html} }

\end{frame}

% -------------------------------
\begin{frame}[fragile]{Truthiness}

{\Large Determining Truthiness:}

\vfill
{\Large \verb+bool(something)+ }

\vfill


\end{frame}

% ---------------------------------------------
\begin{frame}[fragile]{Boolean Expressions}

{\Large \verb+False+ }

\begin{itemize}
  \item \verb+None+
  \item \verb+False+
  \item zero of any numeric type, for example, \verb+ 0, 0L, 0.0, 0j+.
  \item any empty sequence, for example, \verb+ '', (), [] +.
  \item any empty mapping, for example, \verb+{}+.
  \item instances of user-defined classes, if the class defines a
        \verb+__nonzero__() or __len__()+ method, when that method
        returns the integer zero or bool value \verb+False+.
\end{itemize}

\url{http://docs.python.org/library/stdtypes.html}

\end{frame}

% ---------------------------------------------
\begin{frame}[fragile]{Boolean Expressions}

{ \LargeAvoid: }

\vspace{0.1in}
\verb+if xx == True:+

\vfill
{ \LargeUse: }

\vspace{0.1in}
\verb+if xx:+

\vfill
\end{frame}

% ---------------------------------------------
\begin{frame}[fragile]{Boolean Expressions}

{\Large ``Shortcutting''}

\begin{verbatim}
                  if x is false, 
x or y               return y,
                     else return x

                  if x is false,
x and y               return  x
                      else return y

                  if x is false,
not x               return True,
                    else return False 
\end{verbatim}

\end{frame} 

% ---------------------------------------------
\begin{frame}[fragile]{Boolean Expressions}

{\Large Stringing them together}

\begin{verbatim}
 a or b or c or d

a and b and c and d  
\end{verbatim}

{\Large The first value that defines the result is returned}

\vfill
(demo)
\end{frame}


%---------------------------------------------
\begin{frame}[fragile]{Boolean returns}

{\Large From CodingBat}
\vfill
\begin{verbatim}
def sleep_in(weekday, vacation):
    if weekday == True and vacation == False:
        return False
    else:
        return True
\end{verbatim}

\end{frame}


%---------------------------------------------
\begin{frame}[fragile]{Boolean returns}

{\Large From CodingBat}

%\begin{verbatim}
%def makes10(a, b):
%    return a == 10 or b == 10 or a+b == 10
%\end{verbatim}

\begin{verbatim}
def sleep_in(weekday, vacation):
    return not (weekday == True and vacation == False)
\end{verbatim}

or

\begin{verbatim}
def sleep_in(weekday, vacation):
    return (not weekday) or vacation
\end{verbatim}


\end{frame}


% -------------------------------------------
\begin{frame}[fragile]{bools are ints?}

{\Large bool types are subclasses of integer}

\begin{verbatim}
In [1]: True == 1
Out[1]: True

In [2]: False == 0
Out[2]: True  
\end{verbatim}

{\Large It gets weirder! }

\begin{verbatim}
In [6]: 3 + True
Out[6]: 4
\end{verbatim}

(demo)

\end{frame}

%-------------------------------
\begin{frame}[fragile]{Conditional expression}

{\large A common idiom:}
\begin{verbatim}
if something:
    x = a_value
else:
    x = another_value
\end{verbatim}
\vfill
{\large Also, other languages have a ``ternary operator''}\\
\hspace{0.2in}(C family: \verb|result = a > b ? x : y ;|)

\vfill
{ \Large \verb|y = 5 if x > 2 else 3| }

\vfill
{\large PEP 308:}
(http://www.python.org/dev/peps/pep-0308/)

\end{frame}



%-------------------------------
\begin{frame}[fragile]{LAB}

\begin{itemize}
  \item Look up the \verb+%+ operator. What do these do?\\
    \verb| 10 % 7 == 3 | \\
    \verb| 14 % 7 == 0 |
  \item  Write a program that prints the numbers from 1 to 100 inclusive.
But for multiples of three print ``Fizz'' instead of the number and for the
multiples of five print ``Buzz''. For numbers which are multiples of both three
and five print ``FizzBuzz'' instead.

  \item Re-write a couple CodingBat exercises, using a conditional expression


  \item Re-write a couple CodingBat exercises, returning the direct boolean results\\
\end{itemize}

(use whichever you like, or the ones in: \verb|code/codingbat.rst| )

\end{frame}

%-------------------------------
\begin{frame}[fragile]{Lightning Talks}

{\LARGE Lightning Talks:}

\vfill
Ryan Small

\vfill
Catherine Warren 


\end{frame}


%%%%%%%%%%%%%%%%%%%%%%%%%%%%%%%%
\section{Code structure, modules, and namespaces}
%%%%%%%%%%%%%%%%%%%%%%%%%%%%%%%%

% ---------------------------------------------
\begin{frame}[fragile]{Code Structure}

{\Large Python is all about namespaces --  the ``dots'' }

\vfill
\verb|name.another_name|

\vfill
The ``dot'' indicates looking for a name in the namespace of the
given object. It could be:

\begin{itemize}
\item name in a module
\item module in a package
\item attribute of an object
\item method of an object
\end{itemize}

\end{frame}

% ---------------------------------------------
\begin{frame}[fragile]{indenting and blocks}

{\Large  Indenting determines blocks of code }

\vfill
\begin{verbatim}
something:
    some code
    some more code
    another block:
        code in 
        that block
\end{verbatim}

\vfill
{\Large But you need the colon too...}

\end{frame}

% ---------------------------------------------
\begin{frame}[fragile]{indenting and blocks}

{\Large  You can put a one-liner after the colon:}

\vfill
\begin{verbatim}
In [167]: x = 12

In [168]: if x > 4: print x
12
\end{verbatim}

\vfill
{\Large Only do this if it makes it more readable...}

\end{frame}


\begin{frame}[fragile]{Spaces and Tabs}

{\Large  An indent can be:}
\begin{itemize}
  \item Any number of spaces
  \item A tab
  \item tabs and spaces:
    \begin{itemize}
      \item A tab is eight spaces (always!)
      \item Are they eight in your editor?
    \end{itemize}
\end{itemize}

\vfill
{\LARGE Always use four spaces -- really!}

\vfill
(PEP 8)

\end{frame}


% ---------------------------------------------
\begin{frame}[fragile]{Spaces Elsewhere}

{\Large  Other than indenting -- space doesn't matter}

\vfill
\begin{verbatim}

x = 3*4+12/func(x,y,z)

x = 3*4 + 12 /   func (x,   y, z) 

\end{verbatim}

\vfill
{\Large Choose based on readability/coding style}

\vfill
\center{\LARGE PEP 8}

\end{frame}


% ---------------------------------------------
\begin{frame}[fragile]{Various Brackets}

{\Large Bracket types:}

\begin{itemize}
  \item parentheses \verb+( )+
    \begin{itemize}
      \item tuple literal: \verb+(1,2,3)+
      \item function call: \verb+fun( arg1, arg2 )+
      \item grouping: \verb| (a + b) * c |
    \end{itemize}
  \item square brackets \verb+[ ]+
    \begin{itemize}
      \item list literal: \verb+[1,2,3]+
      \item sequence indexing: \verb+a_string[4]+
    \end{itemize}
  \item curly brackets \verb+{ }+
    \begin{itemize}
      \item dictionary literal: \verb+{"this":3, "that":6}+
      \item (we'll get to those...)
    \end{itemize}
\end{itemize}

\end{frame}


%-----------------------------------
\begin{frame}{modules and packages}

{\Large A module is simply a namespace}

\vfill
{\Large A package is a module with other modules in it}

\vfill
{\Large The code in the module is run when it is imported}

\end{frame}

% ---------------------------------------------
\begin{frame}[fragile]{importing modules}

\begin{verbatim}

import modulename

from modulename import this, that

import modulename as a_new_name
\end{verbatim}

\vfill
(demo)

\end{frame} 

% ---------------------------------------------
\begin{frame}[fragile]{importing from packages}

\begin{verbatim}

import packagename.modulename

from packagename.modulename import this, that

from package import modulename

\end{verbatim}
\vfill
(demo)

\vfill
\url{http://effbot.org/zone/import-confusion.htm}

\end{frame} 

% ---------------------------------------------
\begin{frame}[fragile]{importing from packages}

\begin{verbatim}
from modulename import *
\end{verbatim}

\vfill
{\LARGE Don't do this!}
\vfill
{\Large (``Namespaces are one honking great idea...'')}

\vfill
(wxPython and numpy example...)

\vfill
Except \emph{maybe} math module

\vfill
(demo)
\end{frame} 


%------------------------------------
\begin{frame}[fragile]{import}

\vfill
If you don’t know the module name before execution.

\vfill
\begin{verbatim}
__import__(module)
\end{verbatim}

\vfill
where \verb|module| is a Python string.

\vfill
\end{frame}

\begin{frame}[fragile]{modules and packages}

\vfill
{\Large The code in a module is NOT re-run when imported again
 -- it must be explicitly reloaded to be re-run}

\begin{verbatim}
import modulename

reload(modulename)
\end{verbatim}

(demo)

\begin{verbatim}
import sys
print sys.modules
\end{verbatim}
(demo)
\end{frame}


%-------------------------------
\begin{frame}[fragile]{LAB}

{\Large  Experiment with importing different ways:}
\begin{verbatim}
import math
dir(math) # or, in ipython -- math.<tab>
math.sqrt(4)

import math as m
m.sqrt(4)

from math import *
sqrt(4)
\end{verbatim}

\end{frame}

%-------------------------------
\begin{frame}[fragile]{LAB}

{\Large  Experiment with importing different ways:}
\begin{verbatim}
import sys
print sys.path

import os
print os.path
\end{verbatim}
{\Large You wouldn't want to import * those -- check out}
\begin{verbatim}
os.path.split()
os.path.join()
\end{verbatim}

\end{frame}

% ---------------------------------------------
\begin{frame}[fragile]{Lightning Talks}

\vfill
{\LARGE Lightning talks next Week:}

\vfill
{\Large
Nate Flagg

\vfill
Duane Wright

\vfill
Josh Rakita

\vfill
Anyone want a slot?
}
\vfill

\end{frame}


%-------------------------------
\begin{frame}[fragile]{Homework}

Recommended Reading:
\begin{itemize}
  \item Think Python: Chapters 8, 9, 10, 11, 12
  \item String methods: \url{http://docs.python.org/library/stdtypes.html#string-methods}
  \item Dive Into Python: Chapter 3
\end{itemize}

Do:
\begin{itemize}
    \item The problem in \verb|week-02/homework.rst| 
    \item Six more CodingBat exercises.
    \item LPTHW: for extra practice with the concepts -- some of:
    \begin{description}
        \item[strings:] ex5, ex6, ex7, ex8, ex9, ex10
        \item[raw\_input(), sys.argv:] ex12, ex13, ex14 (needed for files)
    \end{description}    
\end{itemize}

\vfill
(and any labs you didn't finish in class)

\end{frame}

\end{document}

 
