\documentclass{beamer}
%\usepackage[latin1]{inputenc}
\usetheme{Warsaw}
\title[Intro to Python: Week 1]{Introduction  to Python\\
Persistence / Serialization}
\author{Christopher Barker}
\institute{UW Continuing Education}
\date{December 10, 2013}

\usepackage{listings}
\usepackage{hyperref}

\begin{document}

% ---------------------------------------------
\begin{frame}
  \titlepage
\end{frame}

% ---------------------------------------------
\begin{frame}
\frametitle{Table of Contents}
%\tableofcontents[currentsection]
  \tableofcontents
\end{frame}


\section{Review/Questions}

% ---------------------------------------------
\begin{frame}{Review of Previous Class}

{\Large
\begin{itemize}
  \item Decorators
  \item Context Managers
  \item Packaging
\end{itemize}
}

\end{frame}


% ---------------------------------------------
\begin{frame}{Lightning Talks}

\vfill
{\LARGE Lightning talks today:}

\vfill
{\Large
\vfill
Andrew Bae

\vfill
Travis  Grizzel 

\vfill
Adam Leblanc

}
\vfill

\end{frame}


% ---------------------------------------------
\begin{frame}{Projects}

  \vfill
  {\Large Due Dec 14th, 11:59pm PST }

  \vfill
  {\Large Email them to me}

  \vfill
  {\Large Questions?}

  \vfill
  {\large Note: private github project -- should I make it public?}

\end{frame}

%-------------------------------
\begin{frame}{Evaluations}

\vfill
{\Large UWPCE wants you to fill out course evaluations...}

\vfill
{\Large I need a volunteer:
\begin{itemize}
  \item Collect the evals...
  \item Mail them in to UWPCE
\end{itemize}
}

\vfill
{\Large pLease fill them out during the class period}

\vfill
{\large Yes, I do have a handful of \#2 pencils...}

\end{frame}


%##############################
\section{Web Development Class}

\begin{frame}[fragile]{Internet Programming in Python}

\vfill
{\LARGE Internet Programming in Python }

\vfill
{\LARGE Cris Ewing}

\end{frame}

%-------------------------------
\begin{frame}{Lightning Talks}

{\LARGE Lightning Talks:}

\vfill
{\Large Andrew Bae }

\vfill
{\Large Travis  Grizzel }

\vfill

\end{frame}

% ----------------------------------------------
\begin{frame}[fragile]{Side note:}

{\Large How do you spell switch/case in Python?}

\vfill
{\hspace{0.2in} Hint: NOT with a lot of \verb`elif`s}

\vfill
{\Large Use a dict:}

\vfill
{\Large Put the values to switch on in the keys:}

\vfill
{\Large Functions to call in values:}

\vfill
demo: sample code (\verb|switch_case.py|)
\end{frame} 



% ---------------------------------------------
\begin{frame}{This Class}

\vfill
{\Large Today is less about concepts}

\vfill
{\Large More about learning to use a given module}

\vfill
{\Large So less talk, more coding}

\end{frame}


% ---------------------------------------------
\begin{frame}[fragile]{Serialization}

\vfill
{\Large I'm focusing on methods available in the Python standard library}

\vfill
{\Large Serialization is the process of putting your potentially complex
(and nested) python data structures into a linear (serial) form .. i.e. a string of bytes.}

\vfill
{\Large The serial form can be saved to a file, pushed over the wire, etc.}

\end{frame} 

% ---------------------------------------------
\begin{frame}[fragile]{Persistence}

\vfill
{\Large Persistence is saving your python data structure(s) to disk -- so they
will persist once the python process is finished.}

\vfill
{\Large Any serial form can provide persistence (by dumping/loading it to/from
a file), but not all persistence mechanisms are serial (i.e RDBMS)}


\vfill
\url{http://wiki.python.org/moin/PersistenceTools}
\end{frame} 


\section{Python Specific Formats}

% ---------------------------------------------
\begin{frame}[fragile]{Python Literals}

\vfill
{\Large Putting plain old python literals in your file}

\vfill
{\Large Gives a nice, human-editable form for config files, etc.}

\vfill
{\Large Don't use for untrusted sources!!!}

\end{frame} 

% ---------------------------------------------
\begin{frame}[fragile]{Python Literals}

\vfill
{\Large Good for basic python types.}
(can work for your own classes, too -- if you write a good \verb|__repr__|)

\vfill
{\Large In theory, \verb|repr()| always gives a form that can be re-constructed.}

\vfill
{\Large Often \verb|str()| form works too.}

\vfill
{\Large \verb|pprint| (pretty print) module can make it easier to read.}

\end{frame} 

% ---------------------------------------------
\begin{frame}[fragile]{Python Literal Example}

{\small
\begin{verbatim}
# a list of dicts
data = [{'this':5, 'that':4}, {'spam':7, 'eggs':3.4}]

In [51]: s = repr(data) # save a string version:

In [52]: data2 = eval(s) # re-construct with eval:

In [53]: data2 == data # they are equal
Out[53]: True

In [54]: data is data2 # but not the same object
Out[54]: False
\end{verbatim}
}
You can save the string to a file and even use \verb|import|
\end{frame} 

% ---------------------------------------------
\begin{frame}[fragile]{pretty print}

{\small
\begin{verbatim}
In [69]: import pprint

In [71]: repr(data)
Out[71]: "[{'this': 5, 'that': 4}, {'eggs': 3.4, 'spam': 7}, {'foo': 86, 'bar': 4.5}, {'fun': 43, 'baz': 6.5}]"

In [72]: s = pprint.pformat(data)

In [73]: print s
[{'that': 4, 'this': 5},
 {'eggs': 3.4, 'spam': 7},
 {'bar': 4.5, 'foo': 86},
 {'baz': 6.5, 'fun': 43}]
\end{verbatim}
}
\end{frame} 

% ---------------------------------------------
\begin{frame}[fragile]{Pickle}

\vfill
{\Large Pickle is a binary format for python objects}

\vfill
{\Large You can essentially dump any python object to disk (or string, or socket, or...}

\vfill
{\Large \verb|cPickle| is faster than pickle, but
can't be customized -- you usually want \verb|cPickle|} 

\vfill
\url{http://docs.python.org/library/pickle.html}
\end{frame} 


% ---------------------------------------------
\begin{frame}[fragile]{Pickle}

{\small
\begin{verbatim}
In [87]: import cPickle as pickle
In [83]: data
Out[83]: 
[{'that': 4, 'this': 5},
 {'eggs': 3.4, 'spam': 7},
 {'bar': 4.5, 'foo': 86},
 {'baz': 6.5, 'fun': 43}]
In [84]: pickle.dump(data, open('data.pkl', 'wb'))
In [85]: data2 = pickle.load(open('data.pkl', 'rb'))
In [86]: data2 == data
Out[86]: True
\end{verbatim}
}

\vfill
\url{http://docs.python.org/library/pickle.html}
\end{frame} 

% ---------------------------------------------
\begin{frame}[fragile]{Shelve}

\vfill
{\Large A ``shelf'' is a persistent, dictionary-like object}

\vfill
{\Large The values (not the keys!) can be essentially arbitrary Python
objects (anything picklable)}

\vfill
{\Large NOTE: will not reflect changes in mutable objects without
   re-writing them to the db.}\\
   (or use writeback=True)

\vfill
{\Large If less that 100s of MB -- just use a dict and pickle it.}

\vfill
\url{http://docs.python.org/library/shelve.html}
\end{frame} 

% ---------------------------------------------
\begin{frame}[fragile]{Shelve}

{\Large \verb|shelve| presents a \verb|dict| interface:}
\begin{verbatim}
import shelve

d = shelve.open(filename) 
d[key] = data   # store data at key 
data = d[key]   # retrieve a COPY of data at key 
del d[key]      # delete data stored at key 
flag = d.has_key(key)   # true if the key exists

d.close()       # close it
\end{verbatim}

\vfill
\url{http://docs.python.org/library/shelve.html}
\end{frame} 

%-------------------------------
\begin{frame}[fragile]{LAB}

{\large There are two datasets in the \verb|code| dir:}

\begin{verbatim}
add_book_data.py
add_book_data_flat.py
# load with:
from add_book_data import AddressBook
\end{verbatim}

They have address book data -- one with a nested dict, one "flat"

\begin{itemize}
  \item Write a module that saves the data as python literals in a file\\
        --- and reads it back in
  \item Write a module that saves the data as a pickle in a file\\
        --- and reads it back in
  \item Write a module that saves the data in a shelve\\
        --- and accesses it one by one.
\end{itemize}

\end{frame}

%-------------------------------
\begin{frame}{Lightning Talk}

{\LARGE Lightning Talk:}

\vfill
{\Large Adam Leblanc}

\vfill

\end{frame}


\section{Interchange Formats} 

\begin{frame}[fragile]{INI}

{\Large INI files}

(the old Windows config files)

\begin{verbatim}
[Section1]
int = 15
bool = true
float = 3.1415

[Section2]
int = 32
...
\end{verbatim}
\vfill
{\Large Good for configuration data, etc.}
\end{frame}

\begin{frame}[fragile]{ConfigParser}

{\Large Writing \verb|ini| files:}

\begin{verbatim}
import ConfigParser
config = ConfigParser.ConfigParser()

config.add_section('Section1')
config.set('Section1', 'int', '15')
config.set('Section1', 'bool', 'true')
config.set('Section1', 'float', '3.1415')

# Writing our configuration file to 'example.cfg'
config.write( open('example.cfg', 'wb') )
\end{verbatim}

\vfill
Note: all keys and values are strings
\end{frame}

\begin{frame}[fragile]{ConfigParser}

{\Large Reading \verb|ini| files:}

\begin{verbatim}
>>> config = ConfigParser.ConfigParser()
>>> config.read('example.cfg')
>>> config.sections()
['Section1', 'Section2']

>>> config.get('Section1', 'float')
'3.1415'
>>> config.items('Section1')
[('int', '15'), ('bool', 'true'), ('float', '3.1415')]
\end{verbatim}

\vfill
\url{http://docs.python.org/library/configparser.html}
\end{frame}

\begin{frame}[fragile]{CSV}

{\Large CSV (Comma Separated Values) format is the
most common import and export format for spreadsheets and databases.}

\vfill
{\Large No real standard -- the Python csv package more or less follows MS Excel standard}

(with other "dialects" available)

\vfill
{\Large Can use delimiters other than commas...}\\
(I like tabs better)

\vfill
{\Large Most useful for simple tabular data}

\end{frame}

%----------------------------------
\begin{frame}[fragile]{CSV module}

{\Large Reading \verb|CSV| files:}

\begin{verbatim}
>>> import csv
>>> spamReader = csv.reader( open('eggs.csv', 'rb') )
>>> for row in spamReader:
...     print ', '.join(row)
Spam, Spam, Spam, Spam, Spam, Baked Beans
Spam, Lovely Spam, Wonderful Spam
\end{verbatim}

\vfill
{\verb|csv| module takes care of string quoting, etc. for you}

\vfill
\url{http://docs.python.org/library/csv.html}
\end{frame}



\begin{frame}[fragile]{CSV module}

{\Large Writing \verb|CSV| files:}

\begin{verbatim}
>>> import csv
>>> spamWriter = csv.writer(open('eggs.csv', 'wb'), 
                            quoting=csv.QUOTE_MINIMAL)
>>> spamWriter.writerow(['Spam'] * 5 + ['Baked Beans'])
>>> spamWriter.writerow(['Spam', 'Lovely Spam', 'Wonderful Spam'])
\end{verbatim}

\vfill
{\verb|csv| module takes care of string quoting, etc for you}

\vfill
\url{http://docs.python.org/library/csv.html}
\end{frame}


% ---------------------------------------------
\begin{frame}[fragile]{JSON}

\vfill
{\Large JSON (JavaScript Object Notation) is a subset of JavaScript syntax
        used as a lightweight data interchange format.}

\vfill
{\Large Python module has an interface similar to pickle}

\vfill
{\Large Can handle the standard Python data types}

\vfill
{\Large Specializable encoding/decoding for other types -- but I wouldn't do that!}

\vfill
{\Large Presents a similar interface as \verb|pickle|}

\vfill
\url{http://www.json.org/}\\
\url{http://docs.python.org/library/json.html}
\end{frame} 

% ---------------------------------------------
\begin{frame}[fragile]{Python json module}

{\small
\begin{verbatim}
In [94]: s = json.dumps(data)

Out[95]: '[{"this": 5, "that": 4}, {"eggs": 3.4, "spam": 7},
           {"foo": 86, "bar": 4.5}, {"fun": 43, "baz": 6.5}]'
    # looks a lot like python literals...
In [96]: data2 = json.loads(s)

Out[97]: 
[{u'that': 4, u'this': 5},
 {u'eggs': 3.4, u'spam': 7},
...
In [98]: data2 == data
Out[98]: True # they are the same

\end{verbatim}
}
(also \verb|json.dump() and json.load()| for files)
\vfill
\url{http://docs.python.org/library/json.html}
\end{frame} 



% ---------------------------------------------
\begin{frame}[fragile]{XML}

\vfill
{\Large XML is a standardized version of SGML, designed for use as a data
        storage / interchange format.}

\vfill
{\Large NOTE: HTML is also SGML, and modern versions conform to the XML standard.}

\end{frame} 

% ---------------------------------------------
\begin{frame}[fragile]{XML in the python std lib}


\vfill
{\Large \verb|xml.dom|: }

\vfill
{\Large \verb|xml.sax|: }

\vfill
{\Large \verb|xml.parsers.expat|: }

\vfill
{\Large \verb|xml.etree|: }

\url{http://docs.python.org/library/xml.etree.elementtree.html}

\end{frame} 

% ---------------------------------------------
\begin{frame}[fragile]{elementtree}


\vfill
{\Large  The Element type is a flexible container object, designed to store
hierarchical data structures in memory.}

\vfill
{\Large  Essentially an in-memory XML -- can be read from / written-to XML}

\vfill
{\Large an \verb`ElementTree` is an entire XML doc}

\vfill
{\Large an \verb`Element` is a node in that tree}

\vfill
\url{http://docs.python.org/library/xml.etree.elementtree.html}

\end{frame} 

%-------------------------------
\begin{frame}[fragile]{LAB}

\begin{verbatim}
# load with:
from add_book_data import AddressBook
\end{verbatim}

They have address book data -- one with a nested dict, one "flat"

\begin{itemize}
  \item Write a module that saves the data as an INI file\\
        --- and reads it back in
  \item Write a module that saves the data as a CSV file\\
        --- and reads it back in
  \item Write a module that saves the data in JSON\\
        --- and reads it back in
  \item Write a module that saves the data in XML\\
        --- and reads it back in \\
        --- this gets ugly!
\end{itemize}

\end{frame}


\section{DataBases} 

% ---------------------------------------------
\begin{frame}[fragile]{anydbm}

\vfill
{\Large \verb|anydbm| is a generic interface to variants of the DBM database}

\vfill
{\Large Suitable for storing data that fits well into a python dict with strings as both keys and values}

\vfill
{\Large Note: anydbm will use the dbm system that works on your system --
        this may be different on different systems -- so the db files may NOT
        be compatible! \verb|whichdb| will try to figure it out, but it's not
        guaranteed}
\vfill
\url{http://docs.python.org/library/anydbm.html}
\end{frame} 

\begin{frame}[fragile]{anydbm module}

{\Large Writing data:}

\begin{verbatim}
#creating a dbm file:
anydbm.open(filename, 'n') 
\end{verbatim}

{\large flag options are: }
\begin{description}
  \item['r']  Open existing database for reading only (default)
  \item['w']  Open existing database for reading and writing
  \item['c']  Open database for reading and writing, creating it if it doesn’t exist
  \item['n']  Always create a new, empty database, open for reading and writing
\end{description}
\vfill
\url{http://docs.python.org/library/anydbm.html}
\end{frame}

\begin{frame}[fragile]{anydbm module}

{\Large \verb|dbm| provides dict-like interace:}

\begin{verbatim}
db = dbm.open("dbm", "c")

db["first"] = "bruce"
db["second"] = "micheal"
db["third"] = "fred"
db["second"] = "john" #overwrite
db.close()
# read it:
db = dbm.open("dbm", "r")
for key in db.keys():
    print key, db[key]
\end{verbatim}

\vfill
\url{http://docs.python.org/library/anydbm.html}
\end{frame}


% ---------------------------------------------
\begin{frame}[fragile]{sqlite}

\vfill
{\Large SQLite: C library provides a lightweight disk-based single-file database}

\vfill
{\Large Nonstandard variant of the SQL query language}

\vfill
{\Large Very broadly used as as an embedded databases for storing
        application-specific data etc.}

\vfill
Firefox plug-in:\\
\url{https://addons.mozilla.org/en-US/firefox/addon/sqlite-manager/}
\end{frame} 

\begin{frame}[fragile]{python sqlite module}

\vfill
{\Large \verb|sqlite3| Python module wraps C lib -- provides standard DB-API interface}

\vfill
{\Large Allows (and require SQL queries}

\vfill
{\Large Can provide high performance, flexible, portable storage for your app}

\vfill
\url{http://docs.python.org/library/sqlite3.html}
\end{frame} 

%---------------------------------
\begin{frame}[fragile]{python sqlite module}

{\Large Example:}

\begin{verbatim}
import sqlite3
# open a connection to a db file:
conn = sqlite3.connect('example.db')

# or build one in-memory
conn = sqlite3.connect(':memory:')

# create a cursor
c = conn.cursor()
\end{verbatim}

\vfill
\url{http://docs.python.org/library/sqlite3.html}
\end{frame} 


%---------------------------------
\begin{frame}[fragile]{python sqlite module}

{\Large Execute SQL with the cursor:}

\begin{verbatim}
# Create table
c.execute('''CREATE TABLE stocks
             (date text, trans text, symbol text, qty real, price real)''')
# Insert a row of data
c.execute("INSERT INTO stocks VALUES ('2006-01-05','BUY','RHAT',100,35.14)")
# Save (commit) the changes
conn.commit()
# Close the cursor if we are done with it
c.close()
\end{verbatim}

\vfill
\url{http://docs.python.org/library/sqlite3.html}
\end{frame} 

%---------------------------------
\begin{frame}[fragile]{python sqlite module}

{\large \verb|SELECT| creates an cursor that can be iterated:}

\begin{verbatim}
>>> for row in c.execute('SELECT * FROM stocks ORDER BY price'):
        print row

(u'2006-01-05', u'BUY', u'RHAT', 100, 35.14)
(u'2006-03-28', u'BUY', u'IBM', 1000, 45.0)
...
\end{verbatim}

{\large Or you can get the rows one by one or in a list:}

\begin{verbatim}
 c.fetchone()
 c.fetchall()
\end{verbatim}

\end{frame} 


%---------------------------------
\begin{frame}[fragile]{python sqlite module}

\vfill
{\large Good idea to use the DB-API’s parameter substitution:}

\begin{verbatim}
t = (symbol,)
c.execute('SELECT * FROM stocks WHERE symbol=?', t)
print c.fetchone()

# Larger example that inserts many records at a time
purchases = [('2006-03-28', 'BUY', 'IBM', 1000, 45.00),
             ('2006-04-05', 'BUY', 'MSFT', 1000, 72.00),
             ('2006-04-06', 'SELL', 'IBM', 500, 53.00),
            ]
c.executemany('INSERT INTO stocks VALUES (?,?,?,?,?)', purchases)
\end{verbatim}

\vfill
\url{http://xkcd.com/327/}
\end{frame} 

% ---------------------------------------------
\begin{frame}[fragile]{DB-API}

\vfill
{\Large The DB-API spec (PEP 249) is a specification for interaction between Python and Relational Databases.}

\vfill
{\Large Support for a large number of third-party Database drivers:
\begin{itemize}
  \item MySQL
  \item PostgreSQL
  \item Oracle
  \item MSSQL (?)
  \item .....
\end{itemize}
}
\vfill
\url{http://www.python.org/dev/peps/pep-0249}
\end{frame} 


\section{Other Options}

% ---------------------------------------------
\begin{frame}[fragile]{Object-Relation Mappers}

\vfill
{\Large Systems for mapping Python objects to tables}

\vfill
{\Large Saves you writing that glue code (and the SQL) }

\vfill
{\Large Usually deal with mapping to variety of back-ends:\\
 -- test with SQLite, deploy with PostreSQL}

\vfill
{\Large  SQL Alchemy}\\ 
 -- \url{http://www.sqlalchemy.org/}

\vfill
{\Large Django ORM}\\
\url{https://docs.djangoproject.com/en/dev/topics/db/}

\end{frame} 


% ---------------------------------------------
\begin{frame}[fragile]{Object Databases}

{\Large Directly store and retrieve Python Objects.}

\vfill
{\Large Kind of like \verb|shelve|, but more flexible, and give you searching, etc.}

\vfill
{\Large ZODB:}\\
(\url{http://www.zodb.org/})

\vfill
{\Large Durus:}\\
(\url{https://www.mems-exchange.org/software/DurusWorks/})

\end{frame} 

% ---------------------------------------------
\begin{frame}[fragile]{NoSQL}

{\Large Map-Reduce, etc.}

\vfill
{\Large....Big deal for "Big Data": Amazon, Google, etc.}

\vfill
{\Large Document-Oriented Storage}
 
{\large
\begin{itemize}
  \item MongoDB (BSON interface, JSON documents)
  \item CouchDB (Apache):
  \begin{itemize}
    \item  JSON documents
    \item  Javascript querying (MapReduce)  
    \item  HTTP API  
  \end{itemize}
\end{itemize}
}

\end{frame} 

%-------------------------------
\begin{frame}{Evaluations}

{\LARGE I need to submit evaluations to UW}

\vfill
{\LARGE We'll so that now -- then the last LAB}

\end{frame}


\begin{frame}[fragile]{LAB}

\begin{verbatim}
# load with:
from add_book_data import AddressBook
\end{verbatim}

\begin{itemize}
  \item Write a module that saves the data in a dbm datbase\\
        --- and reads it back in
  \item Write a module that saves the data in an SQLItE datbase\\
        --- and reads it back in
        --- helps to know SQL here...
\end{itemize}

\end{frame}

%-------------------------------
\begin{frame}{Homework}

\vfill
{\Large Send me a copy of your project: due next Sunday}

\vfill
{\Large Keep learning about and using Python}
\vfill

\end{frame}




\end{document}

 
